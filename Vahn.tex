\documentclass{article}
\usepackage{color}
\usepackage{soul}
\usepackage{multirow}
\usepackage{alltt}
\usepackage{fontspec}
\usepackage[hidelinks]{hyperref}
\usepackage{float}
\restylefloat{table}
\definecolor{gray}{rgb}{0.84, 0.84, 0.84}
\newcommand{\hlv}[2][gray]{ {\sethlcolor{#1} \hl{#2}} }
\setmainfont{CharisSIL-R.ttf}


\title{Vahn}
\author{S D Blumire}

\begin{document}

\maketitle
\thispagestyle{empty}
\newpage

\tableofcontents
\listoftables
\newpage

\section{How to read}

This book features a number of notational features, designed to increase legibility and ease of understanding when reading. However, in order to fully utilize these you must first learn them.

Firstly, whenever a word in Vahn appears in an English passage, it will have a faint grey highlight as it's background. As such, you can tell the difference between the English word ``Vahn'' and the Vahn word, ``\hlv{vahn}''. Usually these words will not occur in quotation marks, as such when talking about things such as the \hlv{jarehrdeelhaiy}, you will not need to be confused about the origin of this strange, unquoted, undefined word.

Glossing will occur in verbatim sections, so as to allow for easier word spacing. Due to Vahn's oligosynthetic nature, there are 2 glossing forms that will be used. One is a full morphemic gloss, in which every base morpheme is marked in the gloss, they will be separated with a period ``.'' in the gloss. The second if a natural gloss, in which morphemes will be grouped into their larger derived meanings, with a hyphen ``-'' being used to separate major semantic divisions, and a period ``.'' being used to mark smaller, less significant morphemic splits. To demonstrate using the same phrase ``All rivers go through the/a mountain''

\begin{alltt}
Natural:
puhngmoo-ngah zor.wa    torw
mountain-\textsc{meth} river.all go
\end{alltt}

\begin{alltt}
Morphemic:
p.uh.ng.moo.ngah         z.or.wa        tor.w
day.thing.place.stone.in water.path.all path.verb
\end{alltt}

It should be noted that some non standard glossing notations will be used. This is because a number of the grammatical features of Vahn are not easily defined by traditional or commonly accepted linguistics, as such glossing abbreviations have not been predefined. This can be seen in the above, where the "methodative" case has been used. This is not a standard grammatical case, however no studied or documented case (that has been seen in natural language) exists which covers the nuance of \hlv{ngah}, so a new case has been defined.

Another notation feature within this will be showing the morphemic construction of vocabulary. This is done in Vahn through morphemes merging and binding, the following is how it is notated.

\begin{table}[H]
\centering
\begin{tabular}{r|l||r}
Vahn     & Merger & English  \\
\hline\hline
poo      &        & day      \\
suh      & merge  & thing    \\
\hline
puh      &        & sun      \\
ng       & append & place    \\
\hline
puhng    &        & sky      \\
moo      & join   & stone    \\
\hline \hline
puhngmoo &        & mountain
\end{tabular}
\end{table}

There are 3 types of merger in Vahn. A join shows the combining of the initial and final of a morpheme; an append shows the adding of a bound morpheme after a single or merged double unbound morpheme; an infix shows the infixing of a morpheme between the initial and final of a single or merged double unbound morpheme; a join shows the concatenating of 2 unbound morpheme clusters; with the special case double syllable morphemes, when joining to a morpheme or morpheme cluster their final is taken, when being joined by a morpheme or morpheme cluster their initial is taken. The only final thing to note is that when you have multiple morpheme clusters, they are shown by embedding such as in the example that follows, where a secondary level occurs in which \hlv{norngahrar} is formed in order for it to attach to the \hlv{torwn} which is formed in the ``top level''.

\begin{table}[H]
\centering
\begin{tabular}{r|l||r}
Vahn    & Merger     & English     \\
\hline\hline
tor     &            & path        \\
w       & append     & verb        \\
\hline
torw    &            & go (v)      \\
n       & append     & alike       \\
\hline
torwn   &            & going (n)   \\
\hline
\multicolumn{1}{|r|}{nor} && tree \\
\multicolumn{1}{|r|}{ngah} & append & in \\
\hline
\multicolumn{1}{|r|}{norngah} && tree.in \\
\multicolumn{1}{|r|}{rar} & join & man \\
\hline

norngahrar & join    & clothing    \\
\hline\hline
torwnnorngahrar &    & trousers    \\
\end{tabular}
\end{table}

\newpage
\section{History}

\subsection{Real World}

Although the exact point at which Vahn was conceived is lost, it's first public release came out on the 12th of December, 2013.

It has gone through a number of major changes and revisions over it's lifetime, the first of these was the removal of it's tone structures, this led to the ambiguity you see today in the words \hlv{kah} and \hlv{vah}, \hlv{rar} and \hlv{bar}, \hlv{tor} and \hlv{nor}, \hlv{poo} and \hlv{moo}.

The second was the introduction of a pronoun structure. Originally (in old Vahn) one would use ``one'' \hlv{rar} to refer to any person regardless of pronoun. However, this soon proved impractical, so I implemented a pronoun system.

The third was a massive increase in the structure of it's sentence structure. Originally sentences were SVO, with time and location positioned seemingly randomly. However, this was revised to be TLSOV. It was also at this time altered such that noun phrases could not exist without a binding word such as \hlv{chi} or \hlv{chin}. As such, identifying part of speech became a much simpler ordeal.

The fourth was a revamp of it's mathematical system. It gained a much more simple reverse polish style maths system. This led to a revision in ordinal and cardinality.

The final was the addition of temporal phrases, allowing an entire clause to be the time slot of the following clause, and also allowing for simpler constructions of temporally based questions.

\subsection{Constructed World}

To be Written at a Later Date

\newpage
\section{Phonology}

\subsection{Vowels}

Vowels in Vahn primarily serve as the finals (nucleus) in unbound morphemes, however they do appear in a number of the bound morphemes. It's vowel topology may appear fairly scatter-shot, though it does have some degree of symmetry.

\begin{table}[H]
\centering
\begin{tabular}{r|ccccc}
& front & near-front & mid & near-back & back \\
\hline
close & i & & & & u \\
near-close & & ɪ & & & \\
close-mid & e & & & & ɤ \\
mid & & & ɚ & & \\
open-mid & ɛ \& ɛ˞ & & & & ʌ · ɔ \& ɔ˞ \\
near-open & æ & & & & \\
open & & & & ɑ \& ɑ˞ & \\
\end{tabular}
\caption{Vowel Inventory}
\label{Vowel Inventory}
\end{table}

On top of just vowels, there are also 3 diphthongs in Vahn.

\begin{table}[H]
\centering
\begin{tabular}{c|c|c}
ɑɪ & ɔɪ & eɪ
\end{tabular}
\caption{Diphthong Inventory}
\label{Diphthong Inventory}
\end{table}

\noindent The orthography is as follows:

\begin{table}[H]
\centering
\begin{tabular}{cccccccccccccccccccc}
< & ah & ar & or & oo & uh & eu & ee & ih & ai & eh & ehr & oi & uhr \\
oor & a & w & o & ay & i & >\\
/ & æ & ɑ˞ & ɔ˞ & u & ʌ & ɤ & i & ɪ & ɑɪ & ɛ & ɛ˞ & ɔɪ & ɚ & \\
ɔ˞ & ɑ & u & ɔ & eɪ & i& /
\end{tabular}
\caption{Vowel Orthography}
\label{Vowel Orthography}
\end{table}

\newpage

\subsection{Consonants}

Consonants serve primarily as the initials (onsets) in unbound morphemes, and as the entirety or onset of bound morphemes.

\begin{table}[H]
\centering
\begin{tabular}{r|cccccccccc}
 & \rotatebox{90}{Bilabial} 
 & \rotatebox{90}{Labiodental} 
 & \rotatebox{90}{Dental} 
 & \rotatebox{90}{Alveolar} 
 & \rotatebox{90}{Post-Alveolar}
 & \rotatebox{90}{Palatal} 
 & \rotatebox{90}{Velar} 
 & \rotatebox{90}{Eppiglotal} 
 & \rotatebox{90}{Glottal} \\
 \hline
 Nasal & m̥ · m &&& n̥ · n &&& ŋ \\
 Stop & p · b &&& t · d & && k · g \\
 Sibilant fricative	&&&& s · z & ʃ · ʒ \\
 Non-Sibilant fricative & ɸ · β & f · v & θ · ð &&&& x · ɣ & ʜ & h \\
 Approximant &&&& ɹ̥ · ɹ && j \\
 Flap / Tap &&&& ɾ \\
 Lateral Aproximant &&&l̥ · l
 \end{tabular}
\caption{Consonant Inventory}
\label{Consonant Inventory}
\end{table}

Though the above form the consonant inventory of Vahn, a number of common clusters (affricates or otherwise), occur.

\begin{table}[H]
\centering
\begin{tabular}{c|c|c|c|c|c|c|c|c}
m̥m &
n̥n &
t͡ʃ &
d͡ʒ &
l̥l &
ɹ̥ɹ &
hf &
hv &
hj 
\end{tabular}
\caption{Consonant Clusters and Affricates}
\label{Consonant Clusters and Affricates}
\end{table}

The orthography for consonants is less simple, due to the existence of Vahn's morpheme for ``not'', this modifies the sound of the initial, and is written in the Latin alphabet as a <h> after the consonant.

\noindent The Orthography is as follows:

\begin{table}[H]
\centering
\begin{tabular}{ccccccccccccccccccccccccccccccccccccc}
< & k & kh & r & rh & t & th & p & ph & s & sh & d & dh & f & fh \\
g & gh  & h & hh & j & jh & l & lh & z & zh & v & vh & b & bh & n \\
nh & m & mh & ng & ch & y & > \\
/ & k & x & ɹ & ɹ̥ɹ & t & θ & p & ɸ & s & ʃ & d & ð & f & hf\\
g & ɣ & h & ʜ & d͡ʒ & hj & l & l̥l & z & ʒ & v & hv & b & β & n\\
n̥n & m & m̥m & ŋ & t͡ʃ & j & /
\end{tabular}
\caption{Consonant Orthography}
\label{Consonant Orthography}
\end{table}

Additional to this, /ɾ/ occurs whenever in the orthography the cluster <rw> occurs: It is said as /ɾu/.

In clusters such as /hv/ and /hv/, a very short schwa is often said between the consonants to emphasise the distinction and improve clarity. This same short schwa also occurs when two consonants which cannot be fluidly moved between in the mouth occur next to each other.

When the same pattern occurs twice (such as <ngng>), the sound is germinated. In some dialects a schwa may be inserted between the two in place of germination.

\subsection{Phonotactics}

Phonotactically Vahn follows a fairly simple set of rules, imposed on it due to it's limited morphemic inventory and the phonemic structures associated with each of them. It's structure can be summarised as

\begin{center}
(H)C(L)V(C)*
\end{center}

Where ``H'' is a voiceless nasal or /h(ə)/, ``C'' is any consonant or affricate, ``L'' is a /l/, ``V'' is a vowel, and ``*'' shows that the previous character can be repeated indefinitely.

For Example, with periods (``.'') showing repetitions of the above structure.

\begin{table}[H]
\centering
\begin{tabular}{c|c|c}
 Word & Phonetics & Phonotactics  \\
 \hline
 sarrarngah & sɑ˞ɹɑ˞ŋæ & CV.CV.CV \\
 yavarngngor & jævɑ˞ŋːɔ˞ & CV.CV.CV \\
 klahw & klæw & CLVC \\
 thorkn & θɔ˞kən & CV.CVC \\
 vahkmnngah & vækəmnŋæ & CV.CVCC.CV \\
 vahchitorwn & væt͡ʃɪtɔ˞ɾun & CV.CV.CV.CVC 
\end{tabular}
\caption{Phonotactic Examples}
\label{Phonotactic Examples}
\end{table}

\subsection{Stress}

Stress is morpheme dependant, in that primary stress falls on the last unbound morpheme (or double unbound morpheme cluster) in a word, secondary stress on the second to last, of these clusters, tertiary on the third and so on.

This is demonstrated in the following, the morphemic breakdown inserts hyphens to show bound morphemes, and the direction in which they bind. Asterisks show the binds of semibound morphemes.

\begin{table}[H]
\centering
\begin{tabular}{c|c|c}
 Word & Morphemes & Phonetic Stress \\
 \hline
  sarrarngah & sar.rar.-ngah & ˌsɑ˞ˈɹɑ˞ŋæ  \\
  yavarngngor & ya-.var.-ngor & jæˈvɑ˞ŋːɔ˞ \\
  klahw & k.-l-.ah.-w & ˈklæw \\
  thorkn & t.-h-.or.-k.-n & ˈθɔ˞kən \\
  vahkmnngah & vah.-k.-m.-n.-ngah & ˈvækəmnŋæ \\
  vahchitorwn & vah.*chi*.tor.-w.-n & ˌvæt͡ʃɪˈtɔ˞ɾun
\end{tabular}
\caption{Stress Examples}
\label{Stress Examples}
\end{table}

\newpage

\section{Morphology}

\subsection{Introduction}

Morphology is arguably the most important element of any Oligosynthetic language, and for Vahn this is no different. It is auto-defining that all languages of this type contain a very small number of morphemes, with Vahn this number is 37. These morphemes are broken into 3 Categories: Bound Morphemes, Unbound Morphemes, and Semi-bound Morphemes.

\subsection{Bound Morphemes}

Bound morphemes are the area in which Vahn differs from many other Oligosynthetic languages, in that though they do not bind to other morphemes, they are capable of fusing with each other. This is to say that semantically, their meanings can be combined.

This happens through a portmanteau of the onset of one Bound morpheme and the Nucleus (and coda) of another. The base morphemes are as follows:

\begin{table}[H]
\centering
\begin{tabular}{c|c|c}
 Morpheme & Pronunciation & Meaning \\
 \hline
 kah & kæ & time \\
 rar & rɑ˞ & man/being \\
 tor & tɔ˞ & path \\
 poo & pu & day \\
 suh & sʌ & thing/good/yes\\
 deu & dɤ & male/true \\
 fee & fi & female/false \\
 gih & gɪ & life \\
 hai & hɑɪ & child \\
 jeh & d͡ʒɛ & land \\
 jarehr & d͡ʒɑ˞ːɛ˞ & country/tribe/group \\
 laiy & lɑɪj & hand \\
 zoiy & zɔɪj & water \\
 saruhr & sɑ˞ːɚ & room/container/vehicle \\
 vah & væ & text \\
 bar & bɑ˞ & eye \\
 nor & nɔ˞ & tree \\
 moo & mu & stone \\
 paroor & pɑ˞ːɔ˞ & name \\
\end{tabular}
\caption{Bound Morphemes}
\label{Bound Morphemes}
\end{table}

Of these, all but three behave in the same way. The three exceptions are \hlv{jarehr}, \hlv{paroor}, and \hlv{saruhr}.

Other than these, the same basic mergeing mechanic applies to all of them. The initial of the first, and nucleus of the second combine to create a new word. This can be seen in the following :

\begin{table}[H]
\centering
\begin{tabular}{r|l||r}
Vahn     & Merger & English  \\
\hline\hline
zoiy &       & water      \\
tor & merge  & path    \\
\hline \hline
zor &        & river
\end{tabular}
\caption{Morphemic Formation - zor | River}
\label{Morphemic Formation - zor | River}
\end{table}

\begin{table}[H]
\centering
\begin{tabular}{r|l||r}
Vahn     & Merger & English  \\
\hline\hline
kah &       & time      \\
poo & merge  & day    \\
\hline \hline
koo &        & today
\end{tabular}
\caption{Morphemic Formation - koo | Today}
\label{Morphemic Formation - koo | Today}
\end{table}

\begin{table}[H]
\centering
\begin{tabular}{r|l||r}
Vahn     & Merger & English  \\
\hline\hline
nor &       & tree      \\
rar & merge  & man    \\
\hline \hline
nar &        & sleep (n)
\end{tabular}
\caption{Morphemic Formation - nar | Sleep}
\label{Morphemic Formation - nar | Sleep}
\end{table}

The derivation of the above should be relatively apparent, but to break it down further: Water that is a path is a river. Time that is a day is today. The final one on the list is one of the small number of words in Vahn that draw inspiration from chinese; it constists of the character for ``tree'' and ``man'' as the chinese character for ``rest'' does, and it means ``sleep'' in Vahn.

The order of the morphemes when fusing is important also, it can be described that the first part of the combination usually provides the main semantic meaning, and the second modifies it's properties.

\begin{table}[H]
\centering
\begin{tabular}{r|l||r}
Vahn     & Merger & English  \\
\hline\hline
moo &       & stone \\
zoiy & merge  & water\\
\hline \hline
moiy &        & lava/magma
\end{tabular}
\caption{Morphemic Formation - moiy | Lava/Magma}
\label{Morphemic Formation - moiy | Lava/Magma}
\end{table}

\begin{table}[H]
\centering
\begin{tabular}{r|l||r}
Vahn     & Merger & English  \\
\hline\hline
zoiy &       & water      \\
moo & merge  & stone    \\
\hline \hline
zoo &        & ice
\end{tabular}
\caption{Morphemic Formation - moo | Ice}
\label{Morphemic Formation - moo | Ice}
\end{table}

\begin{table}[H]
\centering
\begin{tabular}{r|l||r}
Vahn     & Merger & English  \\
\hline\hline
vah &       & writing      \\
laiy & merge  & hand    \\
\hline \hline
vaiy &        &  handwriting
\end{tabular}
\caption{Morphemic Formation - vaiy | Handwriting}
\label{Morphemic Formation - vaiy | Handwriting}
\end{table}

\begin{table}[H]
\centering
\begin{tabular}{r|l||r}
Vahn     & Merger & English  \\
\hline\hline
laiy &       & hand      \\
vah & merge  & text    \\
\hline \hline
lah &        & writing hand \\
\end{tabular}
\caption{Morphemic Formation - lah | Writing Hand}
\label{Morphemic Formation - lah | Writing Hand}
\end{table}

There are also the three special case unbound morphemes which function differantly: \hlv{jarehr}, \hlv{saruhr}, and \hlv{paroor}, There three are special, as instead of splitting at the onset and grouping the nucleus with the coda to form the initial and final, the word is instead split at it's center vowel, such that \hlv{jarehr} splits into \hlv{jar-} and \hlv{-ehr}, \hlv{saruhr} splits into \hlv{sar} and \hlv{uhr}, and \hlv{paroor} splits into \hlv{par} and \hlv{oor}. They also do not merge like the others, instead they either append or prepend to other unbound morphemes (on their own, or in a merged cluster). 

Of the three special case unbound morphemes, each half carries a slightly different meaning. \hlv{par} refers to the name of an animate thing, \hlv{oor} to the name of an inanimate thing. \hlv{sar} refers to a building or any object that contains people (such as a vehicle), \hlv{uhr} refers to general containers. \hlv{jar} refers to a specific group, whilst \hlv{ehr} can be either ``group of'' or ``of the group''.

\begin{table}[H]
\centering
\begin{tabular}{r|l||r}
Vahn     & Merger & English  \\
\hline\hline
jarehr &       & country      \\
rar & append  & man    \\
\hline \hline
jarrar &        & citizen \\
\end{tabular}
\caption{Morphemic Formation - jarrar | Citizen}
\label{Morphemic Formation - jarrar | Citizen}
\end{table}

\begin{table}[H]
\centering
\begin{tabular}{r|l||r}
Vahn     & Merger & English  \\
\hline\hline
rar &       & man      \\
jarehr & append  & country    \\
\hline \hline
rarehr &        & politician \\
\end{tabular}
\caption{Morphemic Formation - rarehr | Politician}
\label{Morphemic Formation - rarehr | Politician}
\end{table}

\begin{table}[H]
\centering
\begin{tabular}{r|l||r}
Vahn     & Merger & English  \\
\hline\hline
saruhr &       & room\\
tor & append  & path \\
\hline \hline
sartor &        & door \\
\end{tabular}
\caption{Morphemic Formation - sartor | Door}
\label{Morphemic Formation - sartor | Door}
\end{table}

\begin{table}[H]
\centering
\begin{tabular}{r|l||r}
Vahn     & Merger & English  \\
\hline\hline
tor &       &  path\\
saruhr & corridor & room \\
\hline \hline
toruhr &        & corridor\\
\end{tabular}
\caption{Morphemic Formation - toruhr | Corridor}
\label{Morphemic Formation - toruhr | Corridor}
\end{table}

\subsection{Unbound and ``Semibound'' Morphemes}

In addition to it's inventory of bound morphemes, vahn also features 16 unbound morphemes and 2 ``semibound'' morphemes.


\begin{table}[H]
\centering
\begin{tabular}{c|c|c}
 Morpheme & Pronunciation & Meaning \\
 \hline
-k & k & Broken \\
-h- & & Not \\
-th & θ & Only \\
-wa & wɑ & All \\
-l- & l & Number \\
-m & m & Very \\
-n & n & Alike \\
-l & l & Question \\
-w & u / w & Verb \\
-ngah &  ŋæ & Inside \\
-ngol & ŋɔl & Above \\
-ngor & ŋɔ˞& Beside \\
-ngay & ŋeɪ & Below \\
-ngi & ŋi & Begin \\
-t & t & End \\
chi & t͡ʃɪ & Equal \\
ya & jɑ* (jæ) & Result
\end{tabular}
\caption{Unbound Morphemes}
\label{Unbound Morphemes}
\end{table}

The above morphemes have a number of ways in which they can bind. This is notated above by the positioning of the hypen (-) character against the orthographic representation of the morpheme. Things can either bind by appending (``-*'') such that \hlv{kah} and \hlv{-k} bind to create \hlv{klahk}. Another way of binding is by infixing (``-*-''), such that \hlv{kah} and \hlv{-h-} becomes \hlv{khah} (The bound morpheme is placed between the onset and the nucleus of the unbound morpheme, modifying the pronunciation as necessary). Finally, there are those that do not have hyphens, these can bind either by prepending, \hlv{ya} and \hlv{var} bind by prepending to create \hlv{yavar}; appending, \hlv{gih} and \hlv{ya} bind by appending to create \hlv{gihya}; or can exist without binding, \hlv{chi} can stand on it's own as a word (with a meaning similar to the English ``of'').

\end{document}